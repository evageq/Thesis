% !TEX root = file:///e:/thesis5sem/thesis.tex
\documentclass[14pt, a4paper]{extarticle}

\usepackage{listings}

\usepackage[paper=A4]{typearea}
\usepackage{lipsum}

\usepackage{caption}
\usepackage{graphicx}
\graphicspath{{./images/}}
\DeclareGraphicsExtensions{.jpg,.png}
\usepackage{csvsimple}

\usepackage{amsfonts}
\usepackage{amsmath}

\usepackage[english,russian]{babel}

\usepackage{fontspec} 
\usepackage{unicode-math}
\defaultfontfeatures{Ligatures={TeX},Renderer=Basic}
\setmainfont[Ligatures={TeX,Historic}]{Times New Roman}
\setmonofont{Courier New}
\setmathfont{XITSMath-Regular.otf}
\newfontfamily\cyrillicfonttt[Script=Cyrillic]{Courier New}
\babelfont{sf}{Droid Sans}
\numberwithin{equation}{section}

\usepackage{chngcntr}

\usepackage{xcolor}

\usepackage{array}
\newcommand\ChangeRT[1]{\noalign{\hrule height #1}}

\usepackage{tabularx}
\newcolumntype{s}{>{\raggedright\arraybackslash}X}
%\renewcommand{\tabularxcolumn}[1]{m{#1}} % Вертикальное центрирование текста

\usepackage{indentfirst} %отступ первой строки первого абзаца
\linespread{1.5}

\setlength{\footskip}{1cm}
\usepackage{geometry}
\geometry{left=3cm}
\geometry{right=1cm}
\geometry{top=2cm}
\geometry{bottom=2cm}

\setlength{\parindent}{1.25cm}

\usepackage{enumitem}
\setlist{left=\parindent, labelsep=1cm, itemsep=0pt, topsep=0pt}

\usepackage[final]{pdfpages}

\usepackage{titlesec} % оформление заголовков

\titleformat{\section}[block]
	{\bfseries\fontsize{18pt}{21.6pt}\selectfont}
        {\thesection}
        {1em}{}
\titleformat{name=\section,numberless}[block]
	{\centering\bfseries\fontsize{18pt}{21.6pt}\selectfont}
        {}
        {0em}{}{}
\titlespacing{\section}
 {\parindent}% space at the left
 {0em}% space before
 {10mm}% space after
\titleformat{\subsection}[block]
	{\bfseries\hspace{\parindent}\fontsize{16pt}{19.2pt}\selectfont}
        {\thesubsection}
        {1em}{}

% Отображать только заголовки первого уровня
%\setcounter{tocdepth}{1}

\usepackage{etoolbox}

\usepackage{nameref}

\usepackage{xurl}
\usepackage{hyperref}
\hypersetup{
    colorlinks,
    citecolor=black,
    filecolor=black,
    linkcolor=black,
    urlcolor=black,
    breaklinks=true
}
\urlstyle{same}
       
\usepackage{float}
\usepackage{graphicx,kantlipsum,setspace}

\usepackage{newfloat}
\DeclareCaptionType[name=Листинг, placement=htbp]{listing}

\usepackage{fancyvrb}

\DeclareCaptionLabelSeparator{emdash}{\;\textemdash\;}
\captionsetup[figure]{name={Рисунок}, 
                      labelsep=emdash, 
                      justification=centering, 
                      position=above, 
                      singlelinecheck=off, 
                      font={singlespacing, small, bf},
                      labelfont=bf, 
                      skip=6pt}

\captionsetup[table]{name={Таблица}, 
                     labelsep=emdash, 
                     justification=raggedright, 
                     position=top, 
                     singlelinecheck=off, 
                     font={singlespacing, small, it}, 
                     labelfont=it, 
                     skip=0pt, 
                     margin=0cm}

\captionsetup[lstlisting]{labelsep=emdash, 
                          justification=raggedright, 
                          position=top, 
                          singlelinecheck=off, 
                          font={singlespacing, small, it}, 
                          labelfont=it, 
                          skip=0pt, 
                          margin=0cm}


% Нумерация по разделам
\counterwithin{figure}{section}
\counterwithin{table}{section}
\counterwithin{listing}{section}

\usepackage{ragged2e}
\usepackage{microtype}

\justifying
\tolerance=500
\hyphenpenalty=10000 % отключение переноса
\emergencystretch=3em

\usepackage{setspace}

\usepackage{multirow}

\usepackage[
citestyle=gost-numeric,
style=gost-numeric, 
blockpunct=emdash,
backend=biber,
sorting=none
]{biblatex}

\defcounter{biburlnumpenalty}{3000}
\defcounter{biburlucpenalty}{6000}
\defcounter{biburllcpenalty}{9000}

\DeclareFieldFormat{url}{Режим доступа: #1}
\DeclareFieldFormat{urldate}{(Дата обращения: #1)}
\renewcommand*{\entrysetpunct}{\par\nopunct\!\!}


\defbibheading{bibliography}[\bibname]{%
  \section*{\centering #1}%
  \markboth{#1}{#1}}


\usepackage{pdflscape}
\usepackage{everypage}

\newcommand{\Lpagenumber}{\ifdim\textwidth=\linewidth\else\bgroup
  \dimendef\margin=0 %use \margin instead of \dimen0
  \ifodd\value{page}\margin=\oddsidemargin
  \else\margin=\evensidemargin
  \fi
  \raisebox{\dimexpr-\topmargin-\headheight-\headsep-0.5\linewidth}[0pt][0pt]{%
    \rlap{\hspace{\dimexpr-\margin+\textheight+\footskip}%
    \llap{\rotatebox{90}{\thepage}}}}%
\egroup\fi}
\AddEverypageHook{\Lpagenumber}%

\usepackage{tocloft}
\setlength{\cftsecnumwidth}{0pt}
\setlength{\cftsecindent}{0pt}% Remove indent for \section
\setlength{\cftsubsecindent}{0pt}% Remove indent for \subsection
\setlength{\cftsubsubsecindent}{0pt}% Remove indent for \subsubsection
\setlength{\cftbeforesecskip}{0pt}% Change spacing between sections
\renewcommand{\cftsecaftersnumb}{\hspace{1.5em}}
\renewcommand{\cftsecleader}{\cftdotfill{\cftdotsep}}
\renewcommand{\cftdotsep}{1.25}
\renewcommand{\cftsecfont}{\normalfont}
\renewcommand{\cftsecpagefont}{\normalfont}
\renewcommand{\cfttoctitlefont}{\hfil \bfseries \large}

% Поддержка листингов
\usepackage{listings}
\lstdefinestyle{gost}{
    basicstyle=\ttfamily\fontsize{10pt}{10pt}\linespread{1}\selectfont,
    breakatwhitespace=false,
    breaklines=true,
    keepspaces=true,
    showspaces=false,          
    showstringspaces=false,
    frame=single
}
\lstset{style=gost}

\addbibresource{thesis.bib}

\usepackage{lipsum} 


\begin{document}

\counterwithin{lstlisting}{section}

\pretocmd{\section}{\newpage}{}{}

\def\contentsname{СОДЕРЖАНИЕ}

\pagenumbering{gobble}
\begin{titlepage}
\includepdf[pages=-]{title/ivbo-06-21_гостев}
\end{titlepage}
\tableofcontents

\section*{ВВЕДЕНИЕ}
\pagenumbering{arabic}
\setcounter{page}{4}

%%% Актуальность, Важность исследования
В современном мире, где темпы технологического прогресса 
неуклонно растут,  роль информационных технологий (ИТ) в 
бизнес-сфере становится всё более значимой. 
Интеграция ИТ в бизнес-процессы предприятия способствует 
увеличению общей эффективности и производительности бизнеса. 
Как показывают исследования, "революция ИТ и интернета 
способствует выдающимся результатам в экономике 
бизнес-сектора через обмен информацией с использованием 
интернета и электронных устройств, 
облегчая доступность ведения бизнеса между компаниями на 
глобальном уровне"\cite{mgunda2019impacts}.
Построение качественной сети передачи данных и современных 
серверных служб для предприятия, 
является важной задачей, решение которой позволит улучшить 
эффективность работы предприятия 
и качество обслуживания клиентов.
Это показывает, насколько критично для современных 
предприятий интегрировать современные информационные 
технологии в свои процессы, чтобы оставаться 
конкурентоспособными на рынке.

%%% Объект, предмет, цель
Объектом данного исследования является сеть передачи данных, предприятия, 
а предметом - особенности проектирования и реализации сети передачи данных. 
Целью работы является планирование и реализация качественной, гибкой и 
эффективной сети передачи данных для предприятия, осуществляющего розничную 
торговлю изделиями, применяемыми в медицинских целях, в специализированных магазинах.

%%% Задачи

%%% Методы исследования
В ходе проведения исследования будет использован ряд методов. 
Метод моделирования будет использован для создания и тестирования 
модели сети,  что обеспечит возможность визуализации предполагаемых 
результатов до их реализации. Метод аналогии будет служить инструментом 
сравнения различных моделей сетей,  что позволит определить наиболее 
подходящий вариант для конкретного предприятия. При помощи метода 
классификации будут описаны и рассмотрены различные площадки, 
такие как главный штаб, склады и филиалы, а также соответствующие им сети. 
Метод изучения и анализа литературы обеспечит знакомство 
с существующими решениями и теоретическими знаниями в области 
проектирования сетей передачи данных.



%%% Источники информации
Основными источниками информации 
для данной работы станут научные статьи, 
учебники по теме "Технологии передачи данных", 
включая книгу Олифера "Сети", 
а также документация и руководства по работе 
с модулями Cisco и программой 
Cisco Packet Tracer. Также будет использована 
информация с ресурса Habr, 
где представлены актуальные статьи и обзоры 
по теме сетевых технологий.

%%% Интсрументальные средства
В процессе выполнения работы будет применён ряд 
инструментальных средств. 
Программное обеспечение для моделирования сетей Cisco 
Packet Tracer будет использовано для проектирования 
и тестирования сети. Это обеспечит точное 
воссоздание планируемых сетевых структур и их 
тестирование в контролируемых условиях. 
Для обеспечения эффективного разделения сети на 
подсети будет использован калькулятор IP-сетей. 
Это инструмент позволит оптимизировать процесс 
разбиения сети на подсети, ускорив его и 
снизив вероятность ошибок. Веб-сервисы с сайта 
diagrams.net будут использованы для построения 
топологий сети и различных диаграмм. 
Это обеспечит наглядность представления 
структуры сети и связей между её элементами.

%%% Содержание работы

\section{ПЛАНИРОВАНИЕ И ПРОЕКТИРОВАНИЕ СЕТИ ПЕРЕДАЧИ ДАННЫХ}

\subsection{Определение структуры предприятия}
Предприятие, являющееся объектом проектирования сети, обладает многоуровневой 
комплексной структурой, состоящей из разнообразных подразделений и отделов. 
Визуализация организационного строения компании способствует выявлению и устранению
неэффективных компонентов в её структуре, а также наглядному представлению структурных
особенностей предприятия, что может являться полезным при проектировании сети передачи
данных.


Методолгия ARIS предоставляет удобные и результативные 
способы визуализации и анализа организационной структуры компании. На Рисунках 
\ref{fig:mainDepartamentStructure}-\ref{fig:warehouseStructure} представлены
организационные структуры для каждого объекта предприятия: главного штаба, филиала, точки присутствия и склада. Структуры выполнены в соответствии с методологией ARIS.

\begin{figure}[H]
        \centering
        \includegraphics[scale=0.6]{ARIS_mainDepStructure.png}
        \caption{Организационная структура главного штаба}
        \label{fig:mainDepartamentStructure}
\end{figure}

\begin{figure}[H]
        \centering
        \includegraphics[scale=0.6]{ARIS_filialStructure.png}
        \caption{Организационная структура филиала}
        \label{fig:filialStructure}
\end{figure}

\begin{figure}[H]
        \centering
        \includegraphics[scale=0.6]{ARIS_tpStructure.png}
        \caption{Организационная структура точки присутствия}
        \label{fig:tpStructure}
\end{figure}

\begin{figure}[H]
        \centering
        \includegraphics[scale=0.6]{ARIS_warehouseStructure.png}
        \caption{Организационная структура склада}
        \label{fig:warehouseStructure}
\end{figure}

Для полноценного анализа структуры предприятия необходимо учитывать не 
только численность персонала по отделам, но и количество АРМ, 
соответствующих потребностям сотрудников. В большинстве отделов число 
АРМ равно числу работников, за исключением административно-хозяйственной 
службы с посменной работой и группы складской логистики, где АРМ не требуются. 
Данные по количеству сотрудников и АРМ представлены в Таблицах 
\ref{table:mainDepPopul}-\ref{table:warehousePopul}.

\begin{table}[H]
\centering
\small
\caption{Численность персонала в главном штабе}
\label{table:mainDepPopul}
\begin{tabular}{|m{5cm}|m{3cm}|m{3cm}|}
\hline
\textbf{Название отдела} & \textbf{Количество людей} & \textbf{Количество АРМ} \\
\hline
Руководство предприятия & 1 & 1 \\
\hline
Коммерческий отдел & 3 & 3 \\
\hline
Бухгалтерия & 10 & 10 \\
\hline
Отдел кадров & 20 & 20 \\
\hline
Отдел закупок & 30 & 30 \\
\hline
Отдел продаж & 55 & 55 \\
\hline
Административно-хозяйственная служба & 15 & 2 \\
\hline
ИТ-департамент & 20 & 20 \\
\hline
Профильные отделы & 45 & 45 \\
\hline
Серверный отдел & 1 & 1 \\
\hline
\end{tabular}
\end{table}


\begin{table}[H]
\centering
\small
\caption{Численность персонала в филиале}
\label{table:filialPopul}
\begin{tabular}{|m{5cm}|m{3cm}|m{3cm}|}
\hline
\textbf{Название отдела} & \textbf{Количество людей} & \textbf{Количество АРМ} \\
\hline
Руководство предприятия & 1 & 1 \\
\hline
Бухгалтерия & 5 & 5 \\
\hline
Отдел кадров & 10 & 10 \\
\hline
Отдел закупок & 10 & 10 \\
\hline
Отдел продаж & 10 & 10 \\
\hline
Административно-хозяйственная служба & 5 & 2 \\
\hline
ИТ-департамент & 10 & 10 \\
\hline
Профильные отделы & 19 & 19 \\
\hline
\end{tabular}
\end{table}


\begin{table}[H]
\centering
\small
\caption{Численность персонала в точке присутствия}
\label{table:tpPopul}
\begin{tabular}{|m{5cm}|m{3cm}|m{3cm}|}
\hline
\textbf{Название отдела} & \textbf{Количество людей} & \textbf{Количество АРМ} \\
\hline
Руководство точки присутствия & 1 & 1 \\
\hline
Отдел продаж & 2 & 2 \\
\hline
Профильный отдел & 3 & 3 \\
\hline
\end{tabular}
\end{table}


\begin{table}[H]
\centering
\small
\caption{Численность персонала на складе}
\label{table:warehousePopul}
\begin{tabular}{|m{4cm}|m{3cm}|m{3cm}|}
\hline
\textbf{Название отдела} & \textbf{Количество людей} & \textbf{Количество АРМ} \\
\hline
Руководство склада & 1 & 1 \\
\hline
Группа складской логистики & 6 & 1 \\
\hline
Группа товарного учета & 13 & 13 \\
\hline
Служба качества & 10 & 10 \\
\hline
\end{tabular}
\end{table}


\subsection{Расчёт пропускной способности каналов передачи данных}
В данном пункте следует произвести расчет пропускной способности каналов передачи 
данных с учетом применения трехуровневой архитектуры сети, состоящей 
из уровня доступа, агрегации и ядра. 

Расчеты по количеству портов выполняются следующим образом.
На уровне доступа количество портов соответствует количеству терминалов 
с проводным подключением. 
Чтобы рассчитать количество портов на уровне агрегации, необходимо 
учесть коэффициент перехода от уровня доступа, обусловленный тем, 
что конечные узлы не используют весь канал передачи данных на 
постоянной основе и активные приложения не сильно 
чувствительны к задержкам и потерям. Выяснено, что такой коэффициент равен 0.4,
Для уровня ядра требуется соотношение один к одному для пропускной способности канала.
Также необходимо определить скоростные требования
к портам для пользователей: обычным пользователям выделяется 
100 МБит/с, руководящим должностям выделяется 1 ГБит/с, так как 
бесперебойное и надежное соединение в 
этом случае крайне необходимо, учитывая постоянные 
переговоры, использующие интернет-соединение, серверный отдел 
использует полный предлагаемый трафик в 1 ГБит/с. 

В Таблице \ref{table:mainDepAccessLevel} представлены расчеты портов для уровня доступа 
главного штаба.

\begin{table}[H]
\centering
\small
\caption{Расчет портов уровня доступа для главного штаба}
\label{table:mainDepAccessLevel}
\begin{tabular}{|m{4cm}|m{2.5cm}|m{2.5cm}|m{2.5cm}|m{3cm}|}
\hline
\textbf{Название отдела} & \textbf{Количество АРМ} & \textbf{Требования к скорости, Мбит/с} & \textbf{Кол-во портов FastEthernet} & \textbf{Кол-во портов GigabitEthernet} \\
\hline
Руководство предприятия & 1 & 1000 & 0 & 1 \\
\hline
Коммерческий отдел & 3 & 100 & 3 & 0 \\
\hline
Бухгалтерия & 10 & 100 & 10 & 0 \\
\hline
Отдел кадров & 20 & 100 & 20 & 0 \\
\hline
Отдел закупок & 30 & 100 & 30 & 0 \\
\hline
Отдел продаж & 55 & 100 & 55 & 0 \\
\hline
Административно-хозяйственная служба & 2 & 100 & 2 & 0 \\
\hline
ИТ-департамент & 20 & 100 & 20 & 0 \\
\hline
Профильные отделы & 45 & 100 & 45 & 0 \\
\hline
Серверный отдел & 1 & 1000 & 0 & 1 \\
\hline
Итого & 187 & - & 185 & 2 \\
\hline 
\end{tabular}
\end{table}

Чтобы рассчитать нагрузку на уровне агрегации для 
главного штаба, воспользуемся Формулой \ref{formula:mainDepAggregationLoad}.

\begin{equation}
\begin{aligned}
p = & \; 182 \; \text{узла} \times 100 \;\frac{\text{Мбит}}{\text{с}} \times 0.4 + 1 \times 1000 \;\frac{\text{Мбит}}{\text{с}} \times 0.4\; + \\
        & + 1 \times 1000\;\frac{\text{Мбит}}{\text{с}} \times 1 = 8680\;\frac{\text{Мбит}}{\text{с}}
\end{aligned}
\label{formula:mainDepAggregationLoad}
\end{equation}
               
Таким образом требуется как минимум десять портов типа GigabitEthernet.
Необходимо учесть также резервирование портов на уровне агрегации.
Далее представлен итоговый расчет портов для каждого уровня сети
главного штаба (Таблица \ref{table:mainDepCampusNet}). 

\begin{table}[H]
\centering
\small
\caption{Итоговый расчет портов для кампусной сети главного штаба}
\label{table:mainDepCampusNet}
\begin{tabular}{|m{2cm}|m{4cm}|m{3cm}|m{3.5cm}|}
\hline
\textbf{Уровень} & \textbf{Кол-во портов FastEthernet} & \textbf{Кол-во портов GigabitEthernet} & \textbf{Кол-во портов 10GigabitEthernet} \\
\hline
Доступ & 182 & 2 & 0 \\
\hline
Агрегация & 0 & 10 + 10 & 0 \\
\hline
Ядро & 0 & 0 & 1 + 1 \\
\hline
\end{tabular}
\end{table}


        

В Таблице \ref{table:filialAccessLevel} представлены расчеты портов для уровня доступа 
филиала.

\begin{table}[H]
\centering
\small
\caption{Расчет портов уровня доступа для филиала}
\label{table:filialAccessLevel}
\begin{tabular}{|m{4cm}|m{2.5cm}|m{2.5cm}|m{2.5cm}|m{3cm}|}
\hline
\textbf{Название отдела} & \textbf{Количество АРМ} & \textbf{Требования к скорости, Мбит/с} & \textbf{Кол-во портов FastEthernet} & \textbf{Кол-во портов GigabitEthernet} \\
\hline
Руководство предприятия & 1 & 1000 & 0 & 1 \\
\hline
Бухгалтерия & 5 & 100 & 5 & 0 \\
\hline
Отдел кадров & 10 & 100 & 10 & 0 \\
\hline
Отдел закупок & 10 & 100 & 10 & 0 \\
\hline
Отдел продаж & 10 & 100 & 10 & 0 \\
\hline
Административно-хозяйственная служба & 2 & 100 & 2 & 0 \\
\hline
ИТ-департамент & 10 & 100 & 10 & 0 \\
\hline
Профильные отделы & 19 & 100 & 19 & 0 \\
\hline
Итого & 67 & - & 66 & 1 \\
\hline
\end{tabular}
\end{table}

Чтобы рассчитать нагрузку на уровне агрегации для 
филиала, воспользуемся Формулой \ref{formula:filialAggregationLoad}.

\begin{equation}
\begin{aligned}
p = & \; 67 \; \text{узлов} \times 100\;\frac{\text{Мбит}}{\text{с}} \times 0.4 \; + \\
& + 1 \; \text{узел} \times 1000\;\frac{\text{Мбит}}{\text{с}} \times 0.4 = 3080\;\frac{\text{Мбит}}{\text{с}}
\end{aligned}
\label{formula:filialAggregationLoad}
\end{equation}

Таким образом требуется как минимум четыре порта типа GigabitEthernet.
Необходимо учесть также резервирование портов на уровне агрегации.
Далее представлен итоговый расчет портов для каждого уровня сети
главного штаба (Таблица \ref{table:filialCampusNet}). 

\begin{table}[H]
\centering
\small
\caption{Итоговый расчет портов для кампусной сети филиала}
\label{table:filialCampusNet}
\begin{tabular}{|m{2cm}|m{4cm}|m{3cm}|m{3.5cm}|}
\hline
\textbf{Уровень} & \textbf{Кол-во портов FastEthernet} & \textbf{Кол-во портов GigabitEthernet} \\
\hline
Доступ & 66 & 1 \\
\hline
Агрегация & 0 & 4 + 4 \\
\hline
Ядро & 0 & 4 + 4 \\
\hline
\end{tabular}
\end{table}

В Таблице \ref{table:tpAccessLevel} представлены расчеты портов для уровня доступа 
точки присутствия.

\begin{table}[H]
\centering
\small
\caption{Расчет портов уровня доступа для точки присутствия}
\label{table:tpAccessLevel}
\begin{tabular}{|m{4cm}|m{2.5cm}|m{2.5cm}|m{2.5cm}|m{3cm}|}
\hline
\textbf{Название отдела} & \textbf{Количество АРМ} & \textbf{Требования к скорости, Мбит/с} & \textbf{Кол-во портов FastEthernet} \\
\hline
Руководство точки присутствия & 1 & 100 & 0 \\
\hline
Отдел продаж & 2 & 100 & 2 \\
\hline
Профильный отдел & 3 & 100 & 3 \\
\hline
Итого & 6 & - & 6 \\
\hline
\end{tabular}
\end{table}

Чтобы рассчитать нагрузку на уровне агрегации для 
точки присутствия, воспользуемся Формулой \ref{formula:tpAggregationLoad}.

\begin{equation}
p = 6 \; \text{узлов} \times 100\;\frac{\text{Мбит}}{\text{с}} \times 0.4 \; = 240 \;\frac{\text{Мбит}}{\text{с}}
\label{formula:tpAggregationLoad}
\end{equation}


Таким образом требуется как минимум один порт типа GigabitEthernet.
Стоит отметить, что в точки присутствия ядро сети избыточно и, 
следовательно, не будет использоваться.
В Таблице \ref{table:filialCampusNet} представлен итоговый расчет портов для точки присутствия.

\begin{table}[H]
\centering
\small
\caption{Итоговый расчет портов для кампусной сети филиала}
\label{table:filialCampusNet}
\begin{tabular}{|m{2cm}|m{4cm}|m{3cm}|m{3.5cm}|}
\hline
\textbf{Уровень} & \textbf{Кол-во портов FastEthernet} & \textbf{Кол-во портов GigabitEthernet} \\
\hline
Доступ & 6 & 1 \\
\hline
Агрегация & 0 & 2 \\
\hline
\end{tabular}
\end{table}


В Таблице \ref{table:tpAccessLevel} представлены расчеты портов для уровня доступа 
точки склада.

\begin{table}[H]
\centering
\small
\caption{Расчет портов уровня доступа для склада}
\label{table:warehouseAccessLevel}
\begin{tabular}{|m{3cm}|m{2.5cm}|m{2.5cm}|m{2.5cm}|}
\hline
\textbf{Название отдела} & \textbf{Количество АРМ} & \textbf{Требования к скорости, Мбит/с} & \textbf{Кол-во портов FastEthernet} \\
\hline
Руководство склада & 1 & 100 & 1 \\
\hline
Группа складской логистики & 1 & 100 & 1 \\
\hline
Группа товарного учета & 10 & 100 & 10 \\
\hline
Служба качества & 10 & 100 & 10 \\
\hline
Итого & 22 & - & 22 \\
\hline
\end{tabular}
\end{table}

Чтобы рассчитать нагрузку на уровне агрегации для 
склада, воспользуемся Формулой \ref{formula:warehouseAggregationLoad}.

\begin{equation}
p = 22 \; \text{узла} \times 100\;\frac{\text{Мбит}}{\text{с}} \times 0.4 \; = 880\;\frac{\text{Мбит}}{\text{с}}
\label{formula:warehouseAggregationLoad}
\end{equation}

Таким образом требуется как минимум один порт типа GigabitEthernet.
Стоит отметить, что на складе ядро сети избыточно и, 
следовательно, не будет использоваться.
В Таблице \ref{table:warehouseCampusNet} представлен итоговый расчет портов для склада.


\begin{table}[H]
\centering
\small
\caption{Итоговый расчет портов для кампусной сети склада}
\label{table:warehouseCampusNet}
\begin{tabular}{|m{2cm}|m{4cm}|m{3cm}|m{3.5cm}|}
\hline
\textbf{Уровень} & \textbf{Кол-во портов FastEthernet} & \textbf{Кол-во портов GigabitEthernet} \\
\hline
Доступ & 22 & 0 \\
\hline
Агрегация & 0 & 1 \\
\hline
\end{tabular}
\end{table}

\newpage
\subsection{Прототипирование сети}
В данном шаге требуется создать прототип сети с учетом 
предварительного планирования количество портов, 
политик резервирования, добавления сервера для 
развертывания программной инфраструктуры предприятия, 
трехуровневой архитектуры сети, полученных в предыдущем 
пункте работы. 
Для формирования прототипа используется программное 
обеспечение draw.io и нотация Cisco. Топология главного штаба приведена на Рисунке \ref{fig:mainDepTopo}.


\begin{landscape}
\begin{figure}[H]
        \centering
        \includegraphics[scale=0.2]{topo_mainDep.png}
        \caption{Прототипируемая топология сети главного штаба}
        \label{fig:mainDepTopo}
\end{figure}
\end{landscape}

\begin{landscape}
Спецификация устройств промежуточных устройств прототипа сети главного штаба представлена в Таблице \ref{table:mainDepDevicesSpecs}.

\begin{table}[H]
\centering
\small
\caption{Спецификация промежуточных устройств прототипа главного штаба}
\label{table:mainDepDevicesSpecs}
\begin{tabular}{|m{2.5cm}|m{4cm}|m{2.5cm}|m{4.2cm}|m{5.3cm}|m{4cm}|}
\hline
\textbf{Модель устройств} & \textbf{Имя устройства} & \textbf{Модули расширения} & \textbf{Общее количество портов(с модулями расширения)} & \textbf{Используемые порты(названия, количество)} & \textbf{Свободные порты(названия, количество)} \\
\hline
2960 & 
SW\_1\_L2\_GOSTEV & 
- & 
24 порта FastEthernet, 2 порта GigabitEthernet 
& 2 порта: GigabitEthernet0/1-0/2, 1 порт FastEthernet0/1 
& 24 порта FastEthernet \\
\hline
2960 & 
SW\_2\_L2\_GOSTEV & 
- & 
24 порта FastEthernet, 2 порта GigabitEthernet & 
13 портов: FastEthernet0/1-0/13, 2 порта GigabitEthernet0/1-0/2 & 
11 портов FastEthernet \\
\hline
2960 &
SW\_3\_L2\_GOSTEV &
- &
24 порта FastEthernet, 2 порта GigabitEthernet &
20 портов: FastEthernet0/1-0/20, 2 порта GigabitEthernet0/1-0/2 &
4 порта FastEthernet \\
\hline
2960 &
SW\_4\_L2\_GOSTEV &
- &
24 порта FastEthernet, 2 порта GigabitEthernet &
20 портов: FastEthernet0/1-0/20, 2 порта GigabitEthernet0/1-0/2 &
4 порта FastEthernet \\
\hline
2960 &
SW\_5\_L2\_GOSTEV &
- &
24 порта FastEthernet, 2 порта GigabitEthernet &
10 портов: FastEthernet0/1-0/10, 2 порта GigabitEthernet0/1-0/2 &
14 порта FastEthernet \\
\hline
2960 &
SW\_6\_L2\_GOSTEV &
- &
24 порта FastEthernet, 2 порта GigabitEthernet &
20 портов: FastEthernet0/1-0/20, 2 порта GigabitEthernet0/1-0/2 &
4 портов FastEthernet \\
\hline
2960 &
SW\_7\_L2\_GOSTEV &
- &
24 порта FastEthernet, 2 порта GigabitEthernet &
20 портов: FastEthernet0/1-0/20, 2 порта GigabitEthernet0/1-0/2 &
4 порта FastEthernet \\
\hline
2960 &
SW\_8\_L2\_GOSTEV &
- &
24 порта FastEthernet, 2 порта GigabitEthernet &
17 портов: FastEthernet0/1-0/17, 2 порта GigabitEthernet0/1-0/2 &
7 портов FastEthernet \\
\hline
2960 &
SW\_9\_L2\_GOSTEV &
- &
24 порта FastEthernet, 2 порта GigabitEthernet &
20 портов: FastEthernet0/1-0/20, 2 порта GigabitEthernet0/1-0/2 &
4 порта FastEthernet \\
\end{tabular}
\end{table}
\end{landscape}

\begin{landscape}
\begin{table}[H]
\centering
\small
\caption*{Продолжение Таблицы \ref{table:mainDepDevicesSpecs}}
\begin{tabular}{|m{2.5cm}|m{4cm}|m{2.5cm}|m{4.2cm}|m{5.3cm}|m{4cm}|}
\hline
2960 &
SW\_10\_L2\_GOSTEV &
- &
24 порта FastEthernet, 2 порта GigabitEthernet &
20 портов: FastEthernet0/1-0/20, 2 порта GigabitEthernet0/1-0/2 &
4 порта FastEthernet \\
\hline
2960 &
SW\_11\_L2\_GOSTEV &
- &
24 порта FastEthernet, 2 порта GigabitEthernet &
20 портов: FastEthernet0/1-0/20, 2 порта GigabitEthernet0/1-0/2 &
4 порта FastEthernet \\
\hline
2960 &
SW\_12\_L2\_GOSTEV &
- &
24 порта FastEthernet, 2 порта GigabitEthernet &
5 портов: FastEthernet0/1-0/5, 2 порта GigabitEthernet0/1-0/2 &
19 портов FastEthernet \\
\hline
3650 24 PS &
SW\_1\_L3\_GOSTEV &
- &
24 порта GigabitEthernet, 1 порт 10GigabitEthernet &
14 портов: GigabitEthernet0/1-0/14, 1 порт 10GigabitEthernet &
10 портов GigabitEthernet \\
\hline
3650 24 PS &
SW\_2\_L3\_GOSTEV &
- &
24 порта GigabitEthernet, 1 порт 10GigabitEthernet &
14 портов: GigabitEthernet0/1-0/14, 1 порт 10GigabitEthernet &
10 портов GigabitEthernet \\
\hline
3650 24 PS &
SW\_3\_L3\_GOSTEV &
- &
24 порта GigabitEthernet&
3 порта: GigabitEthernet0/1-0/3 &
21 порт GigabitEthernet \\
\hline
4321 &
R\_1\_L3\_GOSTEV &
- &
3 порта GigabitEthernet &
2 порта: 10GigabitEthernet0/1-0/2, 1 порт GigabitEthernet0/1 &
2 порта GigabitEthernet\\
\hline
\end{tabular}
\end{table}
\end{landscape}

В Таблице \ref{table:mainDepConnectionPlan} указан план по подключению оборудования по портам в главном штабе.

\begin{table}[H]
\small
\centering
\caption{План по подключению конечных устройств в главном здании}
\label{table:mainDepConnectionPlan}
\begin{tabular}{|l|l|l|}
\hline
\textbf{Название устройства}        & \textbf{Порт}            & \textbf{Описание подключения} \\ \hline
\multirow{3}{*}{SW\_1\_L2\_GOSTEV}  & GigabitEthernet0/1       & PC\_1\_1\_GOSTEV              \\ \cline{2-3} 
                                    & GigabitEthernet0/2       & SW\_1\_L3\_GOSTEV             \\ \cline{2-3} 
                                    & FastEthernet0/1          & SW\_2\_L3\_GOSTEV             \\ \hline
\multirow{4}{*}{SW\_2\_L2\_GOSTEV}  & FastEthernet0/1-0/3      & PC\_2\_1-3\_GOSTEV            \\ \cline{2-3} 
                                    & FastEthernet0/4-0/13     & PC\_3\_1-10\_GOSTEV           \\ \cline{2-3} 
                                    & GigabitEthernet0/1       & SW\_1\_L3\_GOSTEV             \\ \cline{2-3} 
                                    & GigabitEthernet0/2       & SW\_2\_L3\_GOSTEV             \\ \hline
\multirow{3}{*}{SW\_3\_L2\_GOSTEV}  & FastEthernet0/1-0/20     & PC\_4\_1-20\_GOSTEV           \\ \cline{2-3} 
                                    & GigabitEthernet0/1       & SW\_1\_L3\_GOSTEV             \\ \cline{2-3} 
                                    & GigabitEthernet0/2       & SW\_2\_L3\_GOSTEV             \\ \hline
\multirow{3}{*}{SW\_4\_L2\_GOSTEV}  & FastEthernet0/1-0/20     & PC\_5\_1-20\_GOSTEV           \\ \cline{2-3} 
                                    & GigabitEthernet0/1       & SW\_1\_L3\_GOSTEV             \\ \cline{2-3} 
                                    & GigabitEthernet0/2       & SW\_2\_L3\_GOSTEV             \\ \hline
\multirow{3}{*}{SW\_5\_L2\_GOSTEV}  & FastEthernet0/1-0/10     & PC\_5\_21-30\_GOSTEV          \\ \cline{2-3} 
                                    & GigabitEthernet0/1       & SW\_1\_L3\_GOSTEV             \\ \cline{2-3} 
                                    & GigabitEthernet0/2       & SW\_2\_L3\_GOSTEV             \\ \hline
\multirow{3}{*}{SW\_6\_L2\_GOSTEV}  & FastEthernet0/1-0/20     & PC\_6\_1-20\_GOSTEV           \\ \cline{2-3} 
                                    & GigabitEthernet0/1       & SW\_1\_L3\_GOSTEV             \\ \cline{2-3} 
                                    & GigabitEthenet0/2        & SW\_2\_L3\_GOSTEV             \\ \hline
\multirow{3}{*}{SW\_7\_L2\_GOSTEV}  & FastEthernet0/1-0/20     & PC\_6\_21-40\_GOSTEV          \\ \cline{2-3} 
                                    & GigabitEthernet0/1       & SW\_1\_L3\_GOSTEV             \\ \cline{2-3} 
                                    & GigabitEthernet0/2       & SW\_2\_L3\_GOSTEV             \\ \hline
\multirow{4}{*}{SW\_8\_L2\_GOSTEV}  & FastEthernet0/1-0/15     & PC\_6\_41-55\_GOSTEV          \\ \cline{2-3} 
                                    & FastEthernet0/16-0/17    & PC\_7\_1-2\_GOSTEV            \\ \cline{2-3} 
                                    & GigabitEtherneet0/1      & SW\_1\_L3\_GOSTEV             \\ \cline{2-3} 
                                    & GigabitEthernet0/2       & SW\_2\_L3\_GOSTEV             \\ \hline
\multirow{3}{*}{SW\_9\_L2\_GOSTEV}  & FastEthernet0/1-0/20     & PC\_8\_1-20\_GOSTEV           \\ \cline{2-3} 
                                    & GigabitEthernet0/1       & SW\_1\_L3\_GOSTEV             \\ \cline{2-3} 
                                    & GigabitEthernet0/2       & SW\_2\_L3\_GOSTEV             \\ \hline
\multirow{3}{*}{SW\_10\_L2\_GOSTEV} & FastEthernet0/1-0/20     & PC\_9\_1-20\_GOSTEV           \\ \cline{2-3} 
                                    & GigabitEthernet0/1       & SW\_1\_L3\_GOSTEV             \\ \cline{2-3} 
                                    & GigabitEthernet0/2       & SW\_2\_L3\_GOSTEV             \\ \hline
\multirow{3}{*}{SW\_11\_L2\_GOSTEV} & FastEthernet0/1-0/20     & PC\_9\_21-40\_GOSTEV          \\ \cline{2-3} 
                                    & GigabitEthernet0/1       & SW\_1\_L3\_GOSTEV             \\ \cline{2-3} 
                                    & GigabitEthernet0/2       & SW\_2\_L3\_GOSTEV             \\ \hline
\multirow{3}{*}{SW\_12\_L2\_GOSTEV} & FastEthernet0/1-0/5      & PC\_9\_41-45\_GOSTEV          \\ \cline{2-3} 
                                    & GigabitEthernet0/1       & SW\_1\_L3\_GOSTEV             \\ \cline{2-3} 
                                    & GigabitEthernet0/2       & SW\_2\_L3\_GOSTEV             \\ \hline
\multirow{4}{*}{SW\_1\_L3\_GOSTEV}  & GigabitEthernet0/1-0/12  & SW\_1-12\_L2\_GOSTEV          \\ \cline{2-3} 
                                    & GigabitEthernet0/13      & SW\_3\_L3\_GOSTEV             \\ \cline{2-3} 
                                    & 10GigabitEthernet0/1     & R1\_L3\_GOSTEV                \\ \cline{2-3} 
                                    & GigabitEthernet0/14      & SW\_2\_L3\_GOSTEV             \\ 
\end{tabular}
\end{table}

\newpage
\begin{table}[H]
\small
\centering
\caption*{Продолжение листинга \ref{table:mainDepConnectionPlan}}
\begin{tabular}{|l|l|l|}
\hline
\multirow{4}{*}{SW\_2\_L3\_GOSTEV}  & GigabitEthernet0/1-0/12  & SW\_1-12\_L2\_GOSTEV          \\ \cline{2-3} 
                                    & GigabitEthernet0/13      & SW\_3\_L3\_GOSTEV             \\ \cline{2-3} 
                                    & 10GigabitEthernet0/1     & R1\_L3\_GOSTEV                \\ \cline{2-3} 
                                    & GigabitEthernet0/14      & SW\_1\_L3\_GOSTEV             \\ \hline
\multirow{3}{*}{SW\_3\_L3\_GOSTEV}  & GigabitEthernet0/1       & S\_1\_GOSTEV                  \\ \cline{2-3} 
                                    & GigabitEthernet0/2       & SW\_1\_L3\_GOSTEV             \\ \cline{2-3} 
                                    & GigabitEthernet0/3       & SW\_2\_L3\_GOSTEV             \\ \cline{2-3}
                                    & GigabitEthernet0/4       & R\_1\_L3\_GOSTEV              \\ \hline
\multirow{2}{*}{R\_1\_L3\_GOSTEV}   & 10GigabitEthernet0/1-0/2 & SW\_1-2\_L3\_GOSTEV           \\ \cline{2-3} 
                                    & GigabitEthernet0/1       & SW\_3\_L3\_GOSTEV             \\ \hline
\end{tabular}
\end{table}

\begin{landscape}
Топология типового филиала приведена на Рисунке \ref{fig:filialTopo}.
\begin{figure}[H]
        \centering
        \includegraphics[scale=0.17]{topo_filial.png}
        \caption{Прототипируемая топология сети филиала}
        \label{fig:filialTopo}
\end{figure}
\end{landscape}

\begin{landscape}
        
Спецификация устройств промежуточных устройств прототипа сети филиала представлена в Таблице \ref{table:filialDevicesSpecs}.

\begin{table}[H]
\centering
\small
\caption{Спецификация промежуточных устройств прототипа филиала}
\label{table:filialDevicesSpecs}
\begin{tabular}{|m{2.5cm}|m{4cm}|m{3cm}|m{4.2cm}|m{5.3cm}|m{4cm}|}
\hline
\textbf{Модель устройств} & \textbf{Имя устройства} & \textbf{Модули расширения} & \textbf{Общее количество портов(с модулями расширения)} & \textbf{Используемые порты(названия, количество)} & \textbf{Свободные порты(названия, количество)} \\
\hline
2960 & 
SW\_1\_L2\_GOSTEV & 
- & 
24 порта FastEthernet, 2 порта GigabitEthernet &
2 порта: GigabitEthernet0/1-0/2, 1 порт FastEthernet0/1 &
23 порта FastEthernet \\
\hline
2960 & 
SW\_2\_L2\_GOSTEV & 
- & 
24 порта FastEthernet, 2 порта GigabitEthernet & 
15 портов: FastEthernet0/1-0/15, 2 порта GigabitEthernet0/1-0/2 & 
9 портов FastEthernet \\
\hline
2960 &
SW\_3\_L2\_GOSTEV &
- &
24 порта FastEthernet, 2 порта GigabitEthernet &
20 портов: FastEthernet0/1-0/20, 2 порта GigabitEthernet0/1-0/2 &
4 порта FastEthernet \\
\hline
2960 &
SW\_4\_L2\_GOSTEV &
- &
24 порта FastEthernet, 2 порта GigabitEthernet &
12 портов: FastEthernet0/1-0/12, 2 порта GigabitEthernet0/1-0/2 &
12 портов FastEthernet \\
\hline
2960 &
SW\_5\_L2\_GOSTEV &
- &
24 порта FastEthernet, 2 порта GigabitEthernet &
19 портов: FastEthernet0/1-0/19, 2 порта GigabitEthernet0/1-0/2 &
5 портов FastEthernet \\
\hline
3650 24 PS &
SW\_1\_L3\_GOSTEV &
- &
24 порта GigabitEthernet &
7 портов: GigabitEthernet0/1-0/7 &
17 портов GigabitEthernet \\
\hline
3650 24 PS &
SW\_2\_L3\_GOSTEV &
- &
24 порта GigabitEthernet &
7 портов: GigabitEthernet0/1-0/7 &
17 портов GigabitEthernet \\
\hline
4321 &
R\_1\_L3\_GOSTEV &
NIM-ES2-4 (4 порта GigabitEthernet) &
6 портов GigabitEthernet &
2 порта: GigabitEthernet0/1-0/2 &
4 порта GigabitEthernet \\
\hline
\end{tabular}
\end{table}
\end{landscape}


План по подключению оборудования по портам в филиале представлен в Таблице \ref{table:filialConnectionPlan}.
\begin{table}[H]
\small
\centering
\caption{План по подключению конечных устройств в филиале}
\label{table:filialConnectionPlan}
\begin{tabular}{|l|l|l|}
\hline
\textbf{Название устройства}       & \textbf{Порт}          & \textbf{Описание подключения} \\ \hline
\multirow{3}{*}{SW\_1\_L2\_GOSTEV} & GigabitEthernet0/1     & PC\_1\_1\_GOSTEV              \\ \cline{2-3} 
                                   & GigabitEthernet0/2     & SW\_1\_L3\_GOSTEV             \\ \cline{2-3} 
                                   & FastEthernet0/1        & SW\_2\_L3\_GOSTEV             \\ \hline
\multirow{4}{*}{SW\_2\_L2\_GOSTEV} & FastEthernet0/1-0/5    & PC\_2\_1-5\_GOSTEV            \\ \cline{2-3} 
                                   & FastEthernet0/6-0/15   & PC\_3\_1-10\_GOSTEV           \\ \cline{2-3} 
                                   & GigabitEthernet0/1     & SW\_1\_L3\_GOSTEV             \\ \cline{2-3} 
                                   & GigabitEthernet0/2     & SW\_2\_L3\_GOSTEV             \\ \hline
\multirow{4}{*}{SW\_3\_L2\_GOSTEV} & FastEthernet0/1-0/10   & PC\_4\_1-10\_GOSTEV           \\ \cline{2-3} 
                                   & FastEthernet0/11-0/20  & PC\_5\_1-10\_GOSTEV           \\ \cline{2-3} 
                                   & GigabitEthernet0/1     & SW\_1\_L3\_GOSTEV             \\ \cline{2-3} 
                                   & GigabitEthernet0/2     & SW\_2\_L3\_GOSTEV             \\ \hline
\multirow{4}{*}{SW\_4\_L2\_GOSTEV} & FastEthernet0/1-0/2    & PC\_6\_2\_GOSTEV              \\ \cline{2-3} 
                                   & FastEthernet0/3-0/12   & PC\_7\_1-10\_GOSTEV           \\ \cline{2-3} 
                                   & GigabitEthernet0/1     & SW\_1\_L3\_GOSTEV             \\ \cline{2-3} 
                                   & GigabitEthernet0/2     & SW\_2\_L3\_GOSTEV             \\ \hline
\multirow{3}{*}{SW\_5\_L2\_GOSTEV} & FastEthernet0/1-0/19   & PC\_8\_1-19\_GOSTEV           \\ \cline{2-3} 
                                   & GigabitEthernet0/1     & SW\_1\_L3\_GOSTEV             \\ \cline{2-3} 
                                   & GigabitEthernet0/2     & SW\_2\_L3\_GOSTEV             \\ \hline
\multirow{3}{*}{SW\_1\_L3\_GOSTEV} & GigabitEthernet0/1-0/5 & SW\_1-5\_L2\_GOSTEV           \\ \cline{2-3} 
                                   & GigabitEthernet0/6     & SW\_2\_L3\_GOSTEV             \\ \cline{2-3} 
                                   & GigabitEthernet0/7     & R\_1\_L3\_GOSTEV              \\ \hline
\multirow{3}{*}{SW\_2\_L3\_GOSTEV} & GigabitEthernet0/1-0/5 & SW\_1-5\_L2\_GOSTEV           \\ \cline{2-3} 
                                   & GigabitEthernet0/6     & SW\_1\_L3\_GOSTEV             \\ \cline{2-3} 
                                   & GigabitEthernet0/7-0/10     & R\_1\_L3\_GOSTEV              \\ \hline
R\_1\_L3\_GOSTEV                   & GigabitEthernet0/1-0/2 & SW\_1-2\_L3\_GOSTEV           \\ \hline
\end{tabular}
\end{table}



Топология типового склада представлена на Рисунке \ref{fig:warehouseTopo}.
\begin{figure}[H]
        \centering
        \includegraphics[scale=0.2]{topo_warehouse.png}
        \caption{Прототипируемая топология сети склада}
        \label{fig:warehouseTopo}
\end{figure}


Спецификация промежуточных устройств прототипа сети склада представлена в Таблице \ref{table:warehouseDevicesSpecs}.

    
\begin{landscape}
\begin{table}[H]
\centering
\small
\caption{Спецификация промежуточных устройств прототипа сети склада}
\label{table:warehouseDevicesSpecs}
\begin{tabular}{|m{2.5cm}|m{4cm}|m{3cm}|m{4.2cm}|m{5.3cm}|m{4cm}|}
\hline
\textbf{Модель устройств} & \textbf{Имя устройства} & \textbf{Модули расширения} & \textbf{Общее количество портов(с модулями расширения)} & \textbf{Используемые порты(названия, количество)} & \textbf{Свободные порты(названия, количество)} \\
\hline
2950T-24 &
SW\_1\_L2\_GOSTEV &
- &
24 порта FastEthernet, 2 порта GigabitEthernet &
22 порта: FastEthernet0/1-0/22, 1 порт GigabitEthernet &
2 порта FastEthernet, 1 порт GigabitEthernet \\
\hline
1941 &
R\_1\_L3\_GOSTEV &
- &
2 порта GigabitEthernet &
1 порт GigabitEthernet0/1 &
1 порт GigabitEthernet \\
\hline
\end{tabular}
\end{table}
\end{landscape}

План по подключению оборудования по портам на складе представлен в Таблице \ref{table:warehouseConnectionPlan}.
\begin{table}[H]
\small
\centering
\caption{План по подключению оборудования по портам на складе}
\label{table:warehouseConnectionPlan}
\begin{tabular}{|l|l|l|}
\hline
\textbf{Название устройства}       & \textbf{Порт}         & \textbf{Описание подключения} \\ \hline
\multirow{5}{*}{SW\_1\_L2\_GOSTEV} & FastEthernet0/1       & PC\_4\_1\_GOSTEV              \\ \cline{2-3} 
                                   & FastEthernet0/2       & PC\_1\_1\_GOSTEV              \\ \cline{2-3} 
                                   & FastEthernet0/3-0/12  & PC\_2\_1-10\_GOSTEV           \\ \cline{2-3} 
                                   & FastEthernet0/13-0/22 & PC\_3\_1-10\_GOSTEV           \\ \cline{2-3} 
                                   & GigabitEthernet0/1    & R\_1\_L3\_GOSTEV              \\ \hline
R\_1\_L3\_GOSTEV                   & GigabitEthernet0/1    & SW\_1\_L2\_GOSTEV             \\ \hline
\end{tabular}
\end{table}






Топология типовой точки присутствия представлена на Рисунке \ref{fig:tpTopo}.
\begin{figure}[H]
        \centering
        \includegraphics[scale=0.2]{topo_tp.png}
        \caption{Прототипируемая топология сети точки присутствия}
        \label{fig:tpTopo}
\end{figure}

Спецификация промежуточных устройст прототипа сети точки присутствия
представлена в Таблице \ref{table:tpDevicesSpecs}
\begin{table}[H]
\small
\centering
\caption{Спецификация промежуточных устройств прототипа сети точки присутствия}
\label{table:tpDevicesSpecs}
\begin{tabular}{|l|l|l|}
\hline
\textbf{Название устройства}       & \textbf{Порт}         & \textbf{Описание подключения} \\ \hline
\multirow{4}{*}{SW\_1\_L2\_GOSTEV} & FastEthernet0/1       & PC\_1\_1\_GOSTEV              \\ \cline{2-3} 
                                   & FastEthernet0/2-0/3   & PC\_2\_1-2\_GOSTEV        \\ \cline{2-3} 
                                   & FastEthernet0/4-0/0/6 & PC\_3\_1-3\_GOSTEV            \\ \cline{2-3} 
                                   & GigabitEthernet0/1    & R\_1\_L3\_GOSTEV              \\ \hline
R\_1\_L3\_GOSTEV                   & GigabitEthernet0/1    & SW\_1\_L2\_GOSTEV             \\ \hline
\end{tabular}
\end{table}


\begin{landscape}
\begin{table}[H]
\centering
\small
\caption{Спецификация промежуточных устройств прототипа филиала}
\begin{tabular}{|m{2.5cm}|m{4cm}|m{3cm}|m{4.2cm}|m{5.3cm}|m{4cm}|}
\hline
\textbf{Модель устройств} & \textbf{Имя устройства} & \textbf{Модули расширения} & \textbf{Общее количество портов(с модулями расширения)} & \textbf{Используемые порты(названия, количество)} & \textbf{Свободные порты(названия, количество)} \\
\hline
2950T-24 &
SW\_1\_L2\_GOSTEV &
- &
24 порта FastEthernet, 2 порта GigabitEthernet &
6 портов: FastEthernet0/1-0/6, 1 порт GigabitEthernet &
18 портов FastEthernet, 1 порт GigabitEthernet \\
\hline
1941 &
R\_1\_L3\_GOSTEV &
- &
2 порта GigabitEthernet &
1 порт GigabitEthernet0/1 &
1 порт GigabitEthernet \\
\hline
\end{tabular}
\end{table}
\end{landscape}



\subsection{Планирование сети уровня 2}

VLAN позволяет нескольким сетям работать практически как одна локальная сеть. Одним из наиболее полезных элементов VLAN является то, что он устраняет задержку в сети, что экономит сетевые ресурсы и повышает эффективность сети. Кроме того, VLAN созданы для обеспечения сегментации и поддержки в таких вопросах, как безопасность, управление сетью и масштабируемость. Трафик также можно легко контролировать с помощью VLAN.



В Таблице \ref{table:mainDepVlan} представлен результат планирования VLAN для сети главного здания.
\begin{table}[H]
\centering
\small
\label{table:mainDepVlan}
\caption{планирование VLAN для главного здания}
\begin{tabular}{|l|l|m{6cm}|}
\hline
\textbf{Индетификатор VLAN} & \textbf{Имя VLAN} & \textbf{Описание} \\
\hline
2 & Leadership & Руководство предприятия \\
\hline
3 & Commercial department & Коммерческий отдел \\
\hline
4 & Bookkeeping & Бухгалтерия \\
\hline
5 & Personnel department & Отдел кадров \\
\hline 
6 & Purchasing department & Отдел закупок \\
\hline 
7 & Sales department & Отдел продаж \\
\hline 
8 & Household services & АХО \\
\hline 
9 & IT department & ИТ-департамент \\
\hline
10 & Specialized department & Профильный отдел \\
\hline
11 & Server & Серверный отдел \\
\hline
100 & Management & Управляющий VLAN для устройств \\
\hline
110-111 & Interconnected & Взаимосвязанные VLAN между уровнем агрегации и ядром \\
\hline
\end{tabular}
\end{table}


Описание конфигурации для последующей настройки VLAN на промеждуточных устройствах
представленно в Таблице \ref{table:mainDepConnectionPlanVlan}.
\begin{table}[H]
\small
\centering
\caption{Планирование виртуальных локальных сетей по портам в главном здании}
\label{table:mainDepConnectionPlanVlan}
\begin{tabular}{|l|l|l|ll|}
\hline
\multirow{2}{*}{\textbf{\begin{tabular}[c]{@{}l@{}}Название \\ устройства\end{tabular}}} & \multirow{2}{*}{\textbf{Порт}} & \multirow{2}{*}{\textbf{\begin{tabular}[c]{@{}l@{}}Описание \\ подключения\end{tabular}}} & \multicolumn{2}{l|}{\textbf{VLAN}} \\ \cline{4-5} 
 &  &  & \multicolumn{1}{l|}{\textbf{access}} & \textbf{trunk} \\ \hline
\multirow{3}{*}{SW\_1\_L2\_GOSTEV} & GigabitEthernet0/1 & PC\_1\_1\_GOSTEV & \multicolumn{1}{l|}{2} &  \\ \cline{2-5} 
 & GigabitEthernet0/2 & SW\_1\_L3\_GOSTEV & \multicolumn{1}{l|}{} & \begin{tabular}[c]{@{}l@{}}2-11,\\ 100,\\ 110-111\end{tabular} \\ \cline{2-5} 
 & FastEthernet0/1 & SW\_2\_L3\_GOSTEV & \multicolumn{1}{l|}{} & \begin{tabular}[c]{@{}l@{}}2-11,\\ 100,\\ 110-111\end{tabular} \\ \hline
\multirow{4}{*}{SW\_2\_L2\_GOSTEV} & FastEthernet0/1-0/3 & PC\_2\_1-3\_GOSTEV & \multicolumn{1}{l|}{3} &  \\ \cline{2-5} 
 & FastEthernet0/4-0/13 & PC\_3\_1-10\_GOSTEV & \multicolumn{1}{l|}{4} &  \\ \cline{2-5} 
 & GigabitEthernet0/1 & SW\_1\_L3\_GOSTEV & \multicolumn{1}{l|}{} & \begin{tabular}[c]{@{}l@{}}2-11,\\ 100,\\ 110-111\end{tabular} \\ \cline{2-5} 
 & GigabitEthernet0/2 & SW\_2\_L3\_GOSTEV & \multicolumn{1}{l|}{} & \begin{tabular}[c]{@{}l@{}}2-11,\\ 100,\\ 110-111\end{tabular} \\ \hline
\multirow{3}{*}{SW\_3\_L2\_GOSTEV} & FastEthernet0/1-0/20 & PC\_4\_1-20\_GOSTEV & \multicolumn{1}{l|}{5} &  \\ \cline{2-5} 
 & GigabitEthernet0/1 & SW\_1\_L3\_GOSTEV & \multicolumn{1}{l|}{} & \begin{tabular}[c]{@{}l@{}}2-11,\\ 100,\\ 110-111\end{tabular} \\ \cline{2-5} 
 & GigabitEthernet0/2 & SW\_2\_L3\_GOSTEV & \multicolumn{1}{l|}{} & \begin{tabular}[c]{@{}l@{}}2-11,\\ 100,\\ 110-111\end{tabular} \\ \hline
\multirow{3}{*}{SW\_4\_L2\_GOSTEV} & FastEthernet0/1-0/20 & PC\_5\_1-20\_GOSTEV & \multicolumn{1}{l|}{6} &  \\ \cline{2-5} 
 & GigabitEthernet0/1 & SW\_1\_L3\_GOSTEV & \multicolumn{1}{l|}{} & \begin{tabular}[c]{@{}l@{}}2-11,\\ 100.\\ 110-111\end{tabular} \\ \cline{2-5} 
 & GigabitEthernet0/2 & SW\_2\_L3\_GOSTEV & \multicolumn{1}{l|}{} & \begin{tabular}[c]{@{}l@{}}2-11,\\ 100,\\ 110-111\end{tabular} \\ \hline
\multirow{3}{*}{SW\_5\_L2\_GOSTEV} & FastEthernet0/1-0/10 & PC\_5\_21-30\_GOSTEV & \multicolumn{1}{l|}{6} &  \\ \cline{2-5} 
 & GigabitEthernet0/1 & SW\_1\_L3\_GOSTEV & \multicolumn{1}{l|}{} & \begin{tabular}[c]{@{}l@{}}2-11,\\ 100,\\ 110-111\end{tabular} \\ \cline{2-5} 
 & GigabitEthernet0/2 & SW\_2\_L3\_GOSTEV & \multicolumn{1}{l|}{} & \begin{tabular}[c]{@{}l@{}}2-11,\\ 100,\\ 110-111\end{tabular} \\ \hline
\multirow{3}{*}{SW\_6\_L2\_GOSTEV} & FastEthernet0/1-0/20 & PC\_6\_1-20\_GOSTEV & \multicolumn{1}{l|}{7} &  \\ \cline{2-5} 
 & GigabitEthernet0/1 & SW\_1\_L3\_GOSTEV & \multicolumn{1}{l|}{} & \begin{tabular}[c]{@{}l@{}}2-11,\\ 100,\\ 110-111\end{tabular} \\ \cline{2-5} 
 & GigabitEthenet0/2 & SW\_2\_L3\_GOSTEV & \multicolumn{1}{l|}{} & \begin{tabular}[c]{@{}l@{}}2-11,\\ 100,\\ 110-111\end{tabular} \\ 
\end{tabular}
\end{table}



\begin{table}[H]
\small
\centering
\caption*{Продолжение Таблицы \ref{table:mainDepConnectionPlanVlan}}
\begin{tabular}{|l|l|l|ll|}
\hline
\multirow{3}{*}{SW\_7\_L2\_GOSTEV} & FastEthernet0/1-0/20 & PC\_6\_21-40\_GOSTEV & \multicolumn{1}{l|}{7} &  \\ \cline{2-5} 
 & GigabitEthernet0/1 & SW\_1\_L3\_GOSTEV & \multicolumn{1}{l|}{} & \begin{tabular}[c]{@{}l@{}}2-11,\\ 100,\\ 110-111\end{tabular} \\ \cline{2-5} 
 & GigabitEthernet0/2 & SW\_2\_L3\_GOSTEV & \multicolumn{1}{l|}{} & \begin{tabular}[c]{@{}l@{}}2-11,\\ 100,\\ 110-111\end{tabular} \\ \hline
\multirow{4}{*}{SW\_8\_L2\_GOSTEV} & FastEthernet0/1-0/15 & PC\_6\_41-55\_GOSTEV & \multicolumn{1}{l|}{7} &  \\ \cline{2-5} 
 & FastEthernet0/16-0/17 & PC\_7\_1-2\_GOSTEV & \multicolumn{1}{l|}{8} &  \\ \cline{2-5} 
 & GigabitEtherneet0/1 & SW\_1\_L3\_GOSTEV & \multicolumn{1}{l|}{} & \begin{tabular}[c]{@{}l@{}}2-11,\\ 100,\\ 110-111\end{tabular} \\ \cline{2-5} 
 & GigabitEthernet0/2 & SW\_2\_L3\_GOSTEV & \multicolumn{1}{l|}{} & \begin{tabular}[c]{@{}l@{}}2-11,\\ 100,\\ 110-111\end{tabular} \\ \hline
\multirow{3}{*}{SW\_9\_L2\_GOSTEV} & FastEthernet0/1-0/20 & PC\_8\_1-20\_GOSTEV & \multicolumn{1}{l|}{9} &  \\ \cline{2-5} 
 & GigabitEthernet0/1 & SW\_1\_L3\_GOSTEV & \multicolumn{1}{l|}{} & \begin{tabular}[c]{@{}l@{}}2-11,\\ 100,\\ 110-111\end{tabular} \\ \cline{2-5} 
 & GigabitEthernet0/2 & SW\_2\_L3\_GOSTEV & \multicolumn{1}{l|}{} & \begin{tabular}[c]{@{}l@{}}2-11,\\ 100,\\ 110-111\end{tabular} \\ \hline
\multirow{3}{*}{SW\_10\_L2\_GOSTEV} & FastEthernet0/1-0/20 & PC\_9\_1-20\_GOSTEV & \multicolumn{1}{l|}{10} &  \\ \cline{2-5} 
 & GigabitEthernet0/1 & SW\_1\_L3\_GOSTEV & \multicolumn{1}{l|}{} & \begin{tabular}[c]{@{}l@{}}2-11,\\ 100,\\ 110-111\end{tabular} \\ \cline{2-5} 
 & GigabitEthernet0/2 & SW\_2\_L3\_GOSTEV & \multicolumn{1}{l|}{} & \begin{tabular}[c]{@{}l@{}}2-11,\\ 100,\\ 110-111\end{tabular} \\ \hline
\multirow{3}{*}{SW\_11\_L2\_GOSTEV} & FastEthernet0/1-0/20 & PC\_9\_21-40\_GOSTEV & \multicolumn{1}{l|}{10} &  \\ \cline{2-5} 
 & GigabitEthernet0/1 & SW\_1\_L3\_GOSTEV & \multicolumn{1}{l|}{} & \begin{tabular}[c]{@{}l@{}}2-11,\\ 100,\\ 110-111\end{tabular} \\ \cline{2-5} 
 & GigabitEthernet0/2 & SW\_2\_L3\_GOSTEV & \multicolumn{1}{l|}{} & \begin{tabular}[c]{@{}l@{}}2-11,\\ 100,\\ 110-111\end{tabular} \\ \hline
\multirow{3}{*}{SW\_12\_L2\_GOSTEV} & FastEthernet0/1-0/5 & PC\_9\_41-45\_GOSTEV & \multicolumn{1}{l|}{10} &  \\ \cline{2-5} 
 & GigabitEthernet0/1 & SW\_1\_L3\_GOSTEV & \multicolumn{1}{l|}{} & \begin{tabular}[c]{@{}l@{}}2-11,\\ 100,\\ 110-111\end{tabular} \\ \cline{2-5} 
 & GigabitEthernet0/2 & SW\_2\_L3\_GOSTEV & \multicolumn{1}{l|}{} & \begin{tabular}[c]{@{}l@{}}2-11,\\ 100,\\ 110-111\end{tabular} \\ 
\end{tabular}
\end{table}

\begin{table}[H]
\small
\centering
\caption*{Продолжение Таблицы \ref{table:mainDepConnectionPlanVlan}}
\begin{tabular}{|l|l|l|ll|}
\hline
\multirow{4}{*}{SW\_1\_L3\_GOSTEV} & \begin{tabular}[c]{@{}l@{}}GigabitEthernet0/1-\\ 0/12\end{tabular} & SW\_1-12\_L2\_GOSTEV & \multicolumn{1}{l|}{} & \begin{tabular}[c]{@{}l@{}}2-11,\\ 100,\\ 110-111\end{tabular} \\ \cline{2-5} 
 & GigabitEthernet0/13 & SW\_3\_L3\_GOSTEV & \multicolumn{1}{l|}{} & \begin{tabular}[c]{@{}l@{}}2-11,\\ 100,\\ 110-111\end{tabular} \\ \cline{2-5} 
 & 10GigabitEthernet0/1 & R1\_L3\_GOSTEV & \multicolumn{1}{l|}{} &  \\ \cline{2-5} 
 & GigabitEthernet0/14 & SW\_2\_L3\_GOSTEV & \multicolumn{1}{l|}{} & \begin{tabular}[c]{@{}l@{}}2-11,\\ 100,\\ 110-111\end{tabular} \\ \hline
\multirow{4}{*}{SW\_2\_L3\_GOSTEV} & \begin{tabular}[c]{@{}l@{}}GigabitEthernet0/1-\\ 0/12\end{tabular} & SW\_1-12\_L2\_GOSTEV & \multicolumn{1}{l|}{} & \begin{tabular}[c]{@{}l@{}}2-11,\\ 100,\\ 110-111\end{tabular} \\ \cline{2-5} 
 & GigabitEthernet0/13 & SW\_3\_L3\_GOSTEV & \multicolumn{1}{l|}{} & \begin{tabular}[c]{@{}l@{}}2-11,\\ 100,\\ 110-111\end{tabular} \\ \cline{2-5} 
 & 10GigabitEthernet0/1 & R1\_L3\_GOSTEV & \multicolumn{1}{l|}{} &  \\ \cline{2-5} 
 & GigabitEthernet0/14 & SW\_1\_L3\_GOSTEV & \multicolumn{1}{l|}{} & \begin{tabular}[c]{@{}l@{}}2-11,\\ 100,\\ 110-111\end{tabular} \\ \hline
\multirow{4}{*}{SW\_3\_L3\_GOSTEV} & GigabitEthernet0/1 & S\_1\_GOSTEV & \multicolumn{1}{l|}{11} &  \\ \cline{2-5} 
 & GigabitEthernet0/2 & SW\_1\_L3\_GOSTEV & \multicolumn{1}{l|}{} & \begin{tabular}[c]{@{}l@{}}2-11,\\ 100,\\ 110-111\end{tabular} \\ \cline{2-5} 
 & GigabitEthernet0/3 & SW\_2\_L3\_GOSTEV & \multicolumn{1}{l|}{} & \begin{tabular}[c]{@{}l@{}}2-11,\\ 100,\\ 110-111\end{tabular} \\ \cline{2-5} 
 & GigabitEthernet0/4 & R\_1\_L3\_GOSTEV & \multicolumn{1}{l|}{} &  \\ \hline
\multirow{2}{*}{R\_1\_L3\_GOSTEV} & \begin{tabular}[c]{@{}l@{}}10GigabitEthernet0/1\\ -0/2\end{tabular} & SW\_1-2\_L3\_GOSTEV & \multicolumn{1}{l|}{} &  \\ \cline{2-5} 
 & GigabitEthernet0/1 & SW\_3\_L3\_GOSTEV & \multicolumn{1}{l|}{} &  \\ \hline
\end{tabular}
\end{table}

В Таблице \ref{table:filialVlan} представлен результат планирования VLAN для сети филиала.
\begin{table}[H]
\centering
\small
\caption{Планирование VLAN для филиала}
\label{table:filialVlan}
\begin{tabular}{|l|l|m{6cm}|}
\hline
\textbf{Индетификатор VLAN} & \textbf{Имя VLAN} & \textbf{Описание} \\
\hline
2 & Leadership & Руководство предприятия \\
\hline
3 & Bookkeeping & Бухгалтерия \\
\hline
4 & Personnel department & Отдел кадров \\
\hline 
5 & Purchasing department & Отдел закупок \\
\hline 
6 & Sales department & Отдел продаж \\
\hline 
7 & Household services & АХО \\
\hline 
8 & IT department & ИТ-департамент \\
\hline
9 & Specialized department & Профильный отдел \\
\hline
100 & Management L2 & Управляющая VLAN для устройств уровня 2 \\
\hline
110-111 & Interconnected & Взаимосвязанные VLAN между уровнем агрегации и ядром \\
\hline
\end{tabular}
\end{table}


Описание конфигурации для последующей настройки VLAN на промежуточных устройствах
представленно в Таблице \ref{table:filialConnectionPlanVlan}.

\begin{table}[H]
\small
\centering
\caption{Планирование виртуальных локальных сетей по портам в филиале}
\label{table:filialConnectionPlanVlan}
\begin{tabular}{|l|l|l|ll|}
\hline
\multirow{2}{*}{\textbf{Название устройства}} & \multirow{2}{*}{\textbf{Порт}} & \multirow{2}{*}{\textbf{Описание подключения}} & \multicolumn{2}{l|}{\textbf{VLAN}} \\ \cline{4-5} 
 &  &  & \multicolumn{1}{l|}{\textbf{access}} & \textbf{trunk} \\ \hline
SW\_1\_L2\_GOSTEV & GigabitEthernet0/1 & PC\_1\_1\_GOSTEV & \multicolumn{1}{l|}{2} &  \\ \hline
 & GigabitEthernet0/2 & SW\_1\_L3\_GOSTEV & \multicolumn{1}{l|}{} & \begin{tabular}[c]{@{}l@{}}2-9,\\ 100,\\ 110-111\end{tabular} \\ \hline
 & FastEthernet0/1 & SW\_2\_L3\_GOSTEV & \multicolumn{1}{l|}{} & \begin{tabular}[c]{@{}l@{}}2-9,\\ 100,\\ 110-111\end{tabular} \\ \hline
\multirow{4}{*}{SW\_2\_L2\_GOSTEV} & FastEthernet0/1-0/5 & PC\_2\_1-5\_GOSTEV & \multicolumn{1}{l|}{3} &  \\ \cline{2-5} 
 & FastEthernet0/6-0/15 & PC\_3\_1-10\_GOSTEV & \multicolumn{1}{l|}{4} &  \\ \cline{2-5} 
 & GigabitEthernet0/1 & SW\_1\_L3\_GOSTEV & \multicolumn{1}{l|}{} & \begin{tabular}[c]{@{}l@{}}2-9,\\ 100,\\ 110-111\end{tabular} \\ \cline{2-5} 
 & GigabitEthernet0/2 & SW\_2\_L3\_GOSTEV & \multicolumn{1}{l|}{} & \begin{tabular}[c]{@{}l@{}}2-9,\\ 100,\\ 110-111\end{tabular} \\ \hline
\multirow{4}{*}{SW\_3\_L2\_GOSTEV} & FastEthernet0/1-0/10 & PC\_4\_1-10\_GOSTEV & \multicolumn{1}{l|}{5} &  \\ \cline{2-5} 
 & FastEthernet0/11-0/20 & PC\_5\_1-10\_GOSTEV & \multicolumn{1}{l|}{6} &  \\ \cline{2-5} 
 & GigabitEthernet0/1 & SW\_1\_L3\_GOSTEV & \multicolumn{1}{l|}{} & \begin{tabular}[c]{@{}l@{}}2-9,\\ 100,\\ 110-111\end{tabular} \\ \cline{2-5} 
 & GigabitEthernet0/2 & SW\_2\_L3\_GOSTEV & \multicolumn{1}{l|}{} & \begin{tabular}[c]{@{}l@{}}2-9,\\ 100,\\ 110-111\end{tabular} \\ \hline
\multirow{4}{*}{SW\_4\_L2\_GOSTEV} & FastEthernet0/1-0/2 & PC\_6\_2\_GOSTEV & \multicolumn{1}{l|}{7} &  \\ \cline{2-5} 
 & FastEthernet0/3-0/12 & PC\_7\_1-10\_GOSTEV & \multicolumn{1}{l|}{8} &  \\ \cline{2-5} 
 & GigabitEthernet0/1 & SW\_1\_L3\_GOSTEV & \multicolumn{1}{l|}{} & \begin{tabular}[c]{@{}l@{}}2-9,\\ 100,\\ 110-111\end{tabular} \\ \cline{2-5} 
 & GigabitEthernet0/2 & SW\_2\_L3\_GOSTEV & \multicolumn{1}{l|}{} & \begin{tabular}[c]{@{}l@{}}2-9,\\ 100,\\ 110-111\end{tabular} \\ \hline
\multirow{3}{*}{SW\_5\_L2\_GOSTEV} & FastEthernet0/1-0/19 & PC\_8\_1-19\_GOSTEV & \multicolumn{1}{l|}{9} &  \\ \cline{2-5} 
 & GigabitEthernet0/1 & SW\_1\_L3\_GOSTEV & \multicolumn{1}{l|}{} & \begin{tabular}[c]{@{}l@{}}2-9,\\ 100,\\ 110-111\end{tabular} \\ \cline{2-5} 
 & GigabitEthernet0/2 & SW\_2\_L3\_GOSTEV & \multicolumn{1}{l|}{} & \begin{tabular}[c]{@{}l@{}}2-9,\\ 100,\\ 110-111\end{tabular} \\ 
\end{tabular}
\end{table}

\begin{table}[H]
\small
\centering
\caption*{Продолжение Таблицы \ref{table:filialConnectionPlanVlan}}
\begin{tabular}{|l|l|l|ll|}
\hline
\multirow{3}{*}{SW\_1\_L3\_GOSTEV} & GigabitEthernet0/1-0/5 & SW\_1-5\_L2\_GOSTEV & \multicolumn{1}{l|}{} & \begin{tabular}[c]{@{}l@{}}2-9,\\ 100,\\ 110-111\end{tabular} \\ \cline{2-5} 
 & GigabitEthernet0/6 & SW\_2\_L3\_GOSTEV & \multicolumn{1}{l|}{} & \begin{tabular}[c]{@{}l@{}}2-9,\\ 100,\\ 110-111\end{tabular} \\ \cline{2-5} 
 & GigabitEthernet0/7 & R\_1\_L3\_GOSTEV & \multicolumn{1}{l|}{} &  \\ \hline
\multirow{3}{*}{SW\_2\_L3\_GOSTEV} & GigabitEthernet0/1-0/5 & SW\_1-5\_L2\_GOSTEV & \multicolumn{1}{l|}{} & \begin{tabular}[c]{@{}l@{}}2-9,\\ 100,\\ 110-111\end{tabular} \\ \cline{2-5} 
 & GigabitEthernet0/6 & SW\_1\_L3\_GOSTEV & \multicolumn{1}{l|}{} & \begin{tabular}[c]{@{}l@{}}2-9,\\ 100,\\ 110-111\end{tabular} \\ \cline{2-5} 
 & GigabitEthernet0/7 & R\_1\_L3\_GOSTEV & \multicolumn{1}{l|}{} &  \\ \hline
R\_1\_L3\_GOSTEV & GigabitEthernet0/1-0/2 & SW\_1-2\_L3\_GOSTEV & \multicolumn{1}{l|}{} &  \\ \hline
\end{tabular}
\end{table}



Планирование VLAN для точки присутствия представлено в Таблице \ref{table:tpVlan}.
\begin{table}[H]
\centering
\small
\caption{Планирование VLAN для точки присутствия}
\label{table:tpVlan}
\begin{tabular}{|l|l|m{6cm}|}
\hline
\textbf{Индетификатор VLAN} & \textbf{Имя VLAN} & \textbf{Описание} \\
\hline
2 & Leadership & Руководство точки присутствия \\
\hline
3 & Sales department & Отдел продаж \\
\hline
4 & Specialized department & Профильный отдел \\
\hline
100 & Management L2 & Управляющая VLAN для устройств уровня 2 \\
\hline
\end{tabular}
\end{table}


\begin{table}[H]
\small
\centering
\caption{Планирование виртуальных локальных сетей по портам в точке присутствия}
\label{table:tpConnectionPlanVlan}
\begin{tabular}{|l|l|l|ll|}
\hline
\multirow{2}{*}{Название устройства} & \multirow{2}{*}{Порт} & \multirow{2}{*}{Описание подключения} & \multicolumn{2}{l|}{VLAN} \\ \cline{4-5} 
 &  &  & \multicolumn{1}{l|}{access} & trunk \\ \hline
\multirow{4}{*}{SW\_1\_L2\_GOSTEV} & FastEthernet0/1 & PC\_1\_1\_GOSTEV & \multicolumn{1}{l|}{2} &  \\ \cline{2-5} 
 & FastEthernet0/2-0/3 & PC\_2\_1-2\_GOSTEV & \multicolumn{1}{l|}{3} &  \\ \cline{2-5} 
 & FastEthernet0/4-0/6 & PC\_3\_1-3\_GOSTEV & \multicolumn{1}{l|}{4} &  \\ \cline{2-5} 
 & GigabitEthernet0/1 & R\_1\_L3\_GOSTEV & \multicolumn{1}{l|}{} & \begin{tabular}[c]{@{}l@{}}2, 3, 4\\ 100\end{tabular} \\ \hline
\end{tabular}
\end{table}





Планирование VLAN для склада представлено в Таблице \ref{table:warehouseVlan}.
\begin{table}[H]
\centering
\small
\caption{Планирование VLAN для склада }
\label{table:warehouseVlan}
\begin{tabular}{|l|l|m{6cm}|}
\hline
\textbf{Индетификатор VLAN} & \textbf{Имя VLAN} & \textbf{Описание} \\
\hline
2 & Leadership & Руководство склада \\
\hline
3 & Logistic & Группа складской логистики \\
\hline
4 & Commodity & Группа товарного учета \\
\hline
5 & Quality & Служба качества \\
\hline
100 & Management L2 & Управляющая VLAN для устройств уровня 2 \\
\hline
\end{tabular}
\end{table}


Планирование виртуальных локальных сетей по портам на складе представлено в Таблице \ref{table:warehouseConnectionPlanVlan}.
\begin{table}[H]
\small
\centering
\caption{Планирование виртуальных локальных сетей по портам на складе}
\label{table:warehouseConnectionPlanVlan}
\begin{tabular}{|l|l|l|ll|}
\hline
\multirow{2}{*}{Название устройства} & \multirow{2}{*}{Порт} & \multirow{2}{*}{Описание подключения} & \multicolumn{2}{l|}{VLAN} \\ \cline{4-5} 
 &  &  & \multicolumn{1}{l|}{access} & trunk \\ \hline
\multirow{4}{*}{SW\_1\_L2\_GOSTEV} & FastEthernet0/1 & PC\_4\_1\_GOSTEV & \multicolumn{1}{l|}{2} &  \\ \cline{2-5} 
 & FastEthernet0/2 & PC\_2\_1-2\_GOSTEV & \multicolumn{1}{l|}{3} &  \\ \cline{2-5} 
 & FastEthernet0/3-0/12 & PC\_3\_1-3\_GOSTEV & \multicolumn{1}{l|}{4} &  \\ \cline{2-5} 
 & FastEthernet0/13-0/22 & PC\_3\_1-3\_GOSTEV & \multicolumn{1}{l|}{5} &  \\ \cline{2-5} 
 & GigabitEthernet0/1 & R\_1\_L3\_GOSTEV & \multicolumn{1}{l|}{} & \begin{tabular}[c]{@{}l@{}}2, 3, 4, 5\\ 100\end{tabular} \\ \hline
\end{tabular}
\end{table}



\subsection{Планирование сети уровня 3}
В данном разделе необходимо спроектировать распределение IP адрессов.
a
Планирование адресации для главного здания представлено в Таблице \ref{table:mainDepIpPlan}.
\begin{table}[H]
\centering
\small
\caption{Планирование адресации для главного задния}
\label{table:mainDepIpPlan}
\begin{tabular}{|m{4cm}|m{3cm}|m{8cm}|}
\hline
\textbf{Сегмент/маска IPсети} & \textbf{Адрес шлюза} & \textbf{Описание сегмента сети} \\
\hline
192.168.2.0/24 & 192.168.2.254 & Сегмент сети, к которому относится руководство предприятия, со шлюзом, расположенным на коммутаторе уровня агрегации
\\ \hline
192.168.3.0/24 & 192.168.3.254 & Сегмент сети, к которому отсносится коммерческий отдел предприятия, со шлюзом, расположенным на коммутаторе уровня агрегации
\\ \hline
192.168.4.0/24 & 192.168.4.254 & Сегмент сети, к которому относится бухгалтерия предприятия, со шлюзом, расположенным на коммутаторе уровня агрегации
\\ \hline
192.168.5.0/24 & 192.168.5.254 & Сегмент сети, к которому относится отдел кадров предприятия, со шлюзом, расположенным на коммутаторе уровня агрегации
\\ \hline
192.168.6.0/24 & 192.168.6.254 & Сегмент сети, к которому относится отдел закупок предприятия, со шлюзом, расположенным на коммутаторе уровня агрегации
\\ \hline
192.168.7.0/24 & 192.168.7.254 & Сегмент сети, к которому относится отдел продаж предприятия, со шлюзом, расположенным на коммутаторе уровня агрегации
\\ \hline
192.168.8.0/24 & 192.168.8.254 & Сегмент сети, к которому относится АХО предприятия, со шлюзом, расположенным на коммутаторе уровня агрегации
\\ \hline
192.168.9.0/24 & 192.168.9.254 & Сегмент сети, к которому относится ИТ-департамент предприятия, со шлюзом, расположенным на коммутаторе уровня агрегации
\\ \hline
192.168.10.0/24 & 192.168.10.254 & Сегмент сети, к которому относится профильный отдел предприятия, со шлюзом, расположенным на коммутаторе уровня агрегации
\\ \hline
192.168.11.0/24 & 192.168.11.254 & Сегмент сети, к которому относится сервер предприятия, со шлюзом, расположенным на коммутаторе уровня агрегации
\\ \hline
192.168.100.0/24 & 192.168.100.254 & Сегмент управляющей сети для устройств уровня 2 со шлюзом, расположенным на коммутаторе уровня агрегации 
\\ \hline
192.168.110.0/30 & - & Сегмент сети для между ядром и уровнем агрегации
\\ \hline
192.168.111.0/30 & - & Сегмент сети для между ядром и уровнем агрегации
\\ \hline
\end{tabular}
\end{table}


План режима распределения IP адресов для главного здания представлен в Таблице \ref{table:mainDepIpDistributionPlan}.

\begin{table}[H]
\centering
\small
\caption{Планирование режима распределения IP адресов для главного задния}
\label{table:mainDepIpDistributionPlan}
\begin{tabular}{|p{4cm}|p{3cm}|p{8cm}|}
\hline
\textbf{Сегмент/Интерфейс IP-сети } & \textbf{Режим распределения} & \textbf{Описание режима распределения} 
\\ \hline
192.168.2.0/24 \newline
192.168.3.0/24 \newline
192.168.4.0/24 \newline
192.168.5.0/24 \newline
192.168.6.0/24 \newline
192.168.7.0/24 \newline
192.168.8.0/24 \newline
192.168.9.0/24 \newline
192.168.10.0/24 \newline
&
DHCP
&
Распределяется коммутатором уровня агрегации. 
\\ \hline
192.168.11.0/24 & Статический & Статически настроенные IP-адреса управления устройством
\\ \hline
192.168.100.0/24 & Статический & Статически настроенные IP-адреса управления устройством 
\\ \hline
192.168.110.0/30 & Статический & Статически настроенные взаимосвязанные IP-адреса
\\ \hline
192.168.111.0/30 & Статический & Статически настроенные взаимосвязанные IP-адреса
\\ \hline
\end{tabular}
\end{table}


Планирование адресации для филиала представлено в Таблице \ref{table:filialIpPlan}.
\begin{table}[H]
\centering
\small
\caption{Планирование адресации для филиала}
\label{table:filialIpPlan}
\begin{tabular}{|m{4cm}|m{3cm}|m{8cm}|}
\hline
\textbf{Сегмент/маска IPсети} & \textbf{Адрес шлюза} & \textbf{Описание сегмента сети} \\
\hline
192.168.2.0/24 & 192.168.2.254 & Сегмент сети, к которому относится руководство предприятия, со шлюзом, расположенным на коммутаторе уровня агрегации
\\ \hline
192.168.3.0/24 & 192.168.3.254 & Сегмент сети, к которому относится бухгалтерия предприятия, со шлюзом, расположенным на коммутаторе уровня агрегации
\\ \hline
192.168.4.0/24 & 192.168.4.254 & Сегмент сети, к которому относится отдел кадров предприятия, со шлюзом, расположенным на коммутаторе уровня агрегации
\\ \hline
192.168.5.0/24 & 192.168.5.254 & Сегмент сети, к которому относится отдел закупок предприятия, со шлюзом, расположенным на коммутаторе уровня агрегации
\\ \hline
192.168.6.0/24 & 192.168.6.254 & Сегмент сети, к которому относится отдел продаж предприятия, со шлюзом, расположенным на коммутаторе уровня агрегации
\\ \hline
192.168.7.0/24 & 192.168.7.254 & Сегмент сети, к которому относится АХО предприятия, со шлюзом, расположенным на коммутаторе уровня агрегации
\\ \hline
192.168.8.0/24 & 192.168.8.254 & Сегмент сети, к которому относится ИТ-департамент предприятия, со шлюзом, расположенным на коммутаторе уровня агрегации
\\ \hline
192.168.9.0/24 & 192.168.9.254 & Сегмент сети, к которому относится профильный отдел предприятия, со шлюзом, расположенным на коммутаторе уровня агрегации
\\ \hline
192.168.100.0/24 & 192.168.100.254 & Сегмент управляющей сети для устройств уровня 2 со шлюзом, расположенным на коммутаторе уровня агрегации 
\\ \hline
192.168.110.0/30 & - & Сегмент сети для между ядром и уровнем агрегации
\\ \hline
192.168.111.0/30 & - & Сегмент сети для между ядром и уровнем агрегации
\\ \hline
\end{tabular}
\end{table}

План режима распределения IP адресов для филиала представлен в Таблице \ref{table:filialDistributionPlan}.

\begin{table}[H]
\centering
\small
\caption{Планирование режима распределения IP адресов для филиала}
\label{table:filialDistributionPlan}
\begin{tabular}{|p{4cm}|p{3cm}|p{8cm}|}
\hline
\textbf{Сегмент/Интерфейс IP-сети } & \textbf{Режим распределения} & \textbf{Описание режима распределения} 
\\ \hline
192.168.2.0/24 \newline
192.168.3.0/24 \newline
192.168.4.0/24 \newline
192.168.5.0/24 \newline
192.168.6.0/24 \newline
192.168.7.0/24 \newline
192.168.8.0/24 \newline
192.168.9.0/24 \newline
&
DHCP
&
Распределяется коммутатором уровня агрегации. 
\\ \hline
192.168.100.0/24 & Статический & Статически настроенные IP-адреса управления устройством 
\\ \hline
192.168.110.0/30 & Статический & Статически настроенные взаимосвязанные IP-адреса
\\ \hline
192.168.111.0/30 & Статический & Статически настроенные взаимосвязанные IP-адреса
\\ \hline
\end{tabular}
\end{table}


Планирование адресации для точки присутствия представлено в Таблице \ref{table:tpIpPlan}.
\begin{table}[H]
\centering
\small
\caption{Планирование адресации для точки присутствия}
\label{table:tpIpPlan}
\begin{tabular}{|m{4cm}|m{3cm}|m{8cm}|}
\hline
\textbf{Сегмент/маска IPсети} & \textbf{Адрес шлюза} & \textbf{Описание сегмента сети} \\
\hline
192.168.2.0/24 & 192.168.2.254 & Сегмент сети, к которому относится руководство предприятия, со шлюзом, расположенным на роутере уровня агрегации
\\ \hline
192.168.3.0/24 & 192.168.3.254 & Сегмент сети, к которому относится отдел продаж предприятия, со шлюзом, расположенным на роутере уровня агрегации
\\ \hline
192.168.4.0/24 & 192.168.4.254 & Сегмент сети, к которому относится профильный отдел предприятия, со шлюзом, расположенным на роутере уровня агрегации
\\ \hline
\end{tabular}
\end{table}



План режима распределения IP адресов для точки присутствия представлен в Таблице \ref{table:tpDistributionPlan}.

\begin{table}[H]
\centering
\small
\caption{Планирование режима распределения IP адресов для точки присутствия}
\label{table:tpDistributionPlan}
\begin{tabular}{|p{4cm}|p{3cm}|p{8cm}|}
\hline
\textbf{Сегмент/Интерфейс IP-сети } & \textbf{Режим распределения} & \textbf{Описание режима распределения} 
\\ \hline
192.168.2.0/24 \newline
192.168.3.0/24 \newline
192.168.4.0/24 \newline
&
DHCP
&
Распределяется маршрутизатором уровня агрегации. 
\\ \hline
192.168.100.0/24 & Статический & Статически настроенные IP-адреса управления устройством 
\\ \hline
\end{tabular}
\end{table}


Планирование адресации для склада представлено в Таблице \ref{table:warehouseIpPlan}.
\begin{table}[H]
\centering
\small
\caption{Планирование адресации для склада}
\label{table:warehouseIpPlan}
\begin{tabular}{|m{4cm}|m{3cm}|m{8cm}|}
\hline
\textbf{Сегмент/маска IPсети} & \textbf{Адрес шлюза} & \textbf{Описание сегмента сети} \\
\hline
192.168.2.0/24 & 192.168.2.254 & Сегмент сети, к которому относится руководство предприятия, со шлюзом, расположенным на маршрутизаторе уровня агрегации
\\ \hline
192.168.3.0/24 & 192.168.3.254 & Сегмент сети, к которому относится бухгалтерия предприятия, со шлюзом, расположенным на маршрутизаторе уровня агрегации
\\ \hline
192.168.4.0/24 & 192.168.4.254 & Сегмент сети, к которому относится отдел кадров предприятия, со шлюзом, расположенным на маршрутизаторе уровня агрегации
\\ \hline
192.168.5.0/24 & 192.168.5.254 & Сегмент сети, к которому относится отдел закупок предприятия, со шлюзом, расположенным на маршрутизаторе уровня агрегации
\\ \hline
192.168.100.0/24 & 192.168.100.254 & Сегмент управляющей сети для устройств уровня 2 со шлюзом, расположенным на маршрутизаторе уровня агрегации 
\\ \hline
\end{tabular}
\end{table}










\subsection{Планирование политик фильтрации трафика}
ACL — это универсальный и мощный механизм фильтрации. С их помощью можно определить на 
кого навешивать определённые политики, а на кого нет.

Политики фильтрации будут определены следующим образом:
\begin{itemize}
        \item Сеть серверов доступна из любого отдела;
        \item Отделы не доступны между собой;
        \item Сеть управления может инициировать соединение
              с любым отделом;
        \item WEB-сервер. Разрешить весь входящий трафик по протоколу HTTP 
              (TCP порт 80) для общего доступа, и одновременно настроить 
              ограниченный доступ для сети управления, 
              разрешая трафик по протоколам Telnet 
              (TCP порт 23) и FTP (TCP порты 20 и 21) только с этого устройства
        \item Файловый сервер. На него должны попадать только резиденты предприятия.
        \item Для DNS сервера нужно открыть порт 53.
        \item В сеть серверов разрешить ICMP-сообщения
\end{itemize}

Стоит отметить, что рекомендуется применять стандартные списки контроля доступа 
ближе к получателям трафика для обеспечения более точного и эффективного управления доступом. 
Такой подход минимизирует риск ненужной фильтрации легитимного трафик,
а расширенные -- ближе к отправителю. В Таблице \ref{} приведено планирование
политик фильтрации трафика.


\subsection{Планирование политик обеспечения качества обслуживания}

QoS, или Quality of Service, это набор технологий на сетевых устройствах, который 
управляет приоритетами трафика и гарантирует высокое качество передачи данных для 
критически важных приложений. Он используется для обеспечения надежной и эффективной 
передачи данных в условиях ограниченной пропускной способности сети.

Определено четыре типа класса трафика: премиальный, золотой, серебряный, бронзовый. 
Трафик iSCSI будет считаться и обрабатываться как премиальный. Трафик золотого класса будет состоять 
из трафика службы управления пользователями и веб-службы. 
Серебряный класс трафика будет содержать трафик службы динамического конфигурирования хостов, а 
бронзовым будет трафик службы доменных имен. Все остальное будет рассматриваться и обрабатываться по модели Best-effort.
На основе этого определена модель поведения для каждого сервиса, результат представлен в Таблице \ref{table:qos}.

\begin{table}[H]
\centering
\small
\caption{Значения DSCP для классов и типов трафика}
\label{table:qos}
\begin{tabular}{|m{3cm}|m{3cm}|m{3cm}|m{3cm}|}
\hline
\textbf{Класс трафика} & \textbf{Тип трафика} & \textbf{Модель поведения} & \textbf{Значение DSCP} \\
\hline
Премиальный & iSCSI & EF & 46 \\
\hline
Золотой & Служба управления пользователями & AF11 & 10 \\
\hline
Серебряный & Веб-сервис & AF12 & 12 \\
\hline
Серебряный & Служба динамического конфигурирования хостов & AF21 & 18 \\
\hline
Бронзовый & Служба доменных имен & AF31 & 26 \\
\hline
\end{tabular}
\end{table}



\subsection{Планирование службы доменных имен}
Система Доменных Имен, является одним из ключевых компонентов современной 
сетевой инфраструктуры предприятия. 
DNS-сервер для предприятия развернут в рамках программного
обеспечения FreeIPA (Free Identity, Policy, Audit), обеспечивая интегрированный
DNS-сервер, облегчая его настройку.
Результаты планирование DNS представлены в Таблице \ref{fig:dns_plan}

\begin{table}[H]
\centering
\small
\label{table:dns_plan}
\caption{Результаты планирования службы доменных имен}
\begin{tabular}{|m{6cm}|m{10cm}|}
\hline
\textbf{Параметр} & \textbf{Значение параметра} \\
\hline
Название зоны &  gostev.test \\
\hline
Запись в файле зоны &  Конфигурируется автоматически программным обеспечением FreeIPA\\
\hline
Название обратной зоны &  11.168.192.in-addr.arpa.\\
\hline
Запись в файле обратной зоны &  Конфигурируется автоматически программным обеспечением FreeIPA\\
\hline
Список серверов, на которые производится пересылка в случае отсутствия данных на сервере &  77.88.8.88, 77.88.8.2\\
\hline
\end{tabular}
\end{table}




\subsection{Планирование политик управления пользователями}
Планирование политик управления пользователями является ключевым элементом для 
обеспечения безопасности и упорядоченности процессов . 
В данном контексте рассматривается развертывание программного обеспечения 
FreeIPA, которое представляет собой интегрированную систему управления идентификацией, 
политиками и информацией о пользователях.

Главные характеристики FreeIPA
\begin{itemize}
        \item Интегрированное решение для управления информацией о безопасности, 
        сочетающее в себе Linux (Fedora), сервер каталогов 389, MIT Kerberos, 
        NTP, DNS, систему сертификатов Dogtag, SSSD и другие компоненты.

        \item Основан на Open source проектах и стандартных протоколах
        \item Сильный акцент на простоте управления и автоматизации задач по установке и настройке.
\end{itemize}



\section{Моделирование сети передачи данных}
В этой главе следует выполнить моделирование сети с использованием соответствующих 
средств моделирования и представить артефакты данного процесса. 
Моделирование сети поделено на два этапа: моделирование сервисов и моделирование сети передачи данных. 

\subsection{Моделирование сервисов}
Моделирование сервисов производится следующим образом: 
создаются две виртуальные машины с использованием выбранного гипервизора – 
одна машина сервер с развернутыми сервисами, вторая машина – клиент.

В качестве сервера используется виртуальная машины под управление Fedora Server 39 \cite{}(https://fedoraproject.org/server/download), 
имя локального пользователя для сервера -- server\_gostev.
В качесте клиента используется Debian 12, имя локального пользователя для клиента -- client\_gostev. 
Результат установки представлен на Рисунке \ref{fig:vm_installation}.

\begin{figure}[H]
        \centering
        \includegraphics[scale=0.6]{vm_installation.png}
        \caption{Результат установки вирутальных машин}
        \label{fig:vm_installation}
\end{figure}




Служба доменных имен развернута в качестве дополнительного пакета программного обеспечения -- FreeIPA.
Конфигурация сервера DNS, в качестве доменных зон и DNS записей представлена на Рисунках \ref{fig:dnszones}-\ref{fig:dnsrecords}. 

\begin{figure}[H]
        \centering
        \includegraphics[scale=0.7]{services/dns/dnszones.png}
        \caption{DNS зоны}
        \label{fig:dnszones}
\end{figure}

\begin{figure}[H]
        \centering
        \includegraphics[scale=0.7]{services/dns/dns_records.png}
        \caption{DNS записи}
        \label{fig:dnsrecords}
\end{figure}

Конфигурация клиента представлена на Рисунке \ref{dns_client_setup}.
\begin{figure}[H]
        \centering
        \includegraphics[scale=1.2]{services/dns/client_dns_setup.png}
        \caption{Конфигурация DNS на клиенте}
        \label{dns_client_setup}
\end{figure}

Тестирование DNS с помощью nslookup представлено на Рисунке \ref{fig:nslookup}.
\begin{figure}[H]
        \centering
        \includegraphics[scale=1]{services/dns/nslookup.png}
        \caption{Тестирование DNS}
        \label{fig:nslookup}
\end{figure}


Далее представлено развертывание службы динамического конфигурирования хостов.
Конфигурация сервера представлена на Рисунке \ref{dhcp_server_cfg}.

\begin{figure}[H]
        \centering
        \includegraphics[scale=0.9]{services/dhcp/servercfg.png}
        \caption{Конфигурация сервера}
        \label{dhcp_server_cfg}
\end{figure}

Конфигурация клиента представлена на Рисунке \ref{dhcp_client_cfg}.

\begin{figure}[H]
        \centering
        \includegraphics[scale=1.2]{services/dhcp/clientcfg.png}
        \caption{Конфигурация клиента}
        \label{dhcp_client_cfg}
\end{figure}


Результат получения клиентом конфигураций представлен на Рисунке \ref{fig:dhcp_proof}.

\begin{figure}[H]
        \centering
        \includegraphics[scale=0.8]{services/dhcp/clientproofdhcp.png}
        \caption{Результат получения клиентом конфигурации}
        \label{fig:dhcp_proof}
\end{figure}


Далее произведено развертывание службы времени -- NTP. 
На Рисунке \ref{fig:ntp_client_status} представлен статус службы на клиенте.
\begin{figure}[H]
        \centering
        \includegraphics[scale=0.7]{services/ntp/client_status.png}
        \caption{Статус службы на клиенте}
        \label{fig:ntp_client_status}
\end{figure}

На Рисунке \ref{fig:ntp_client_status} представлено текущее время на клиенте.
\begin{figure}[H]
        \centering
        \includegraphics[scale=1]{services/ntp/client_time.png}
        \caption{Текущее время на клиенте}
        \label{fig:ntp_client_status}
\end{figure}


Следующим выполнено разворачивание файловой службы, используя iSCSI. 
На Рисунке \ref{fig:iscsi_server_cfg} представлена конфигурация службы на сервере.
\begin{figure}[H]
        \centering
        \includegraphics[scale=1.2]{services/iscsi/server_cfg.png}
        \caption{Конфигурация файловой службы на сервере}
        \label{fig:iscsi_server_cfg}
\end{figure}

На Рисунке \ref{fig:iscsi_server_file_content} представлено содержимое хранилища на сервере.
\begin{figure}[H]
        \centering
        \includegraphics[scale=1.6]{services/iscsi/server_file_content.png}
        \caption{Содержимое хранилища на сервере}
        \label{fig:iscsi_server_file_content}
\end{figure}

На Рисунке \ref{fig:iscsi_connect} представлено доступность хранилища на клиенте.
\begin{figure}[H]
        \centering
        \includegraphics[scale=1]{services/iscsi/connect.png}
        \caption{Проверка доступности хранилища на сервере}
        \label{fig:iscsi_connect}
\end{figure}

На Рисунках \ref{fig:iscsi_write}-\ref{fig:iscsi_read} представлена проверка загрузки и скачивания файлов с хранилища.
\begin{figure}[H]
        \centering
        \includegraphics[scale=0.9]{services/iscsi/write.png}
        \caption{Проверка загрузки на хранилище}
        \label{fig:iscsi_write}
\end{figure}

\begin{figure}[H]
        \centering
        \includegraphics[scale=1.4]{services/iscsi/read.png}
        \caption{Проверка копирования с хранилища}
        \label{fig:iscsi_read}
\end{figure}


Также необходимо развернуть службу управления пользователями в качестве ПО -- FreeIPA.
На Рисунке \ref{fig:ipa_status} представлен статус службы FreeIPA.
\begin{figure}[H]
        \centering
        \includegraphics[scale=1]{services/freeipa/ipa_status.png}
        \caption{Статус службы ipa}
        \label{fig:ipa_status}
\end{figure}

На Рисунке \ref{fig:ipa_users} представлены пользователи домена FreeIPA.
\begin{figure}[H]
        \centering
        \includegraphics[scale=0.65]{services/freeipa/users.png}
        \caption{Пользователи домена FreeIPA}
        \label{fig:ipa_users}
\end{figure}

На Рисунке \ref{fig:ipa_client_logging} представлено логирование клиента в домен FreeIPA.
\begin{figure}[H]
        \centering
        \includegraphics[scale=1.5]{services/freeipa/client_logging.png}
        \caption{Логирование клиента в домен FreeIPA}
        \label{fig:ipa_client_logging}
\end{figure}





\begingroup
\let\itshape\upshape
\sloppy
%\raggedright
%\nocite{*} print everything
\printbibliography[title=список используемых источников]\addcontentsline{toc}{section}{список используемых источников}
\endgroup

\end{document}